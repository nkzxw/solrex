\documentclass[a4paper,10pt]{article}

\usepackage{ifpdf}
\ifpdf
  \usepackage[pdftex]{graphicx}
  \graphicspath{{images/}}
\else
  \RequirePackage[dvipdfm, CJKbookmarks, bookmarks=true, bookmarksnumbered=true%
                unicode,%
             colorlinks,%
         citecolor=blue,%
             hyperindex,%
       plainpages=false,%
      pdfstartview=FitH]{hyperref}
  \AtBeginDvi{\special{pdf:tounicode UTF8-UCS2}}
  \usepackage[dvipdfm]{graphicx}
  \graphicspath{{images/}}
  \DeclareGraphicsExtensions{.eps}
\fi

%\RequirePackage{CJKutf8,CJKnumb,CJKulem}
\RequirePackage{CJKutf8,CJKnumb}
\RequirePackage{color,verbatim,cite}
\RequirePackage{texnames,makeidx,indentfirst}
\RequirePackage{amsmath,amssymb,amsfonts,bm,manfnt}
\RequirePackage{fancyhdr,titlesec,datetime}
\RequirePackage{wasysym,longtable,multirow,bigstrut}
\usepackage[section]{placeins}
\usepackage[left=2.54cm,right=2.54cm,top=3.3cm,bottom=2.6cm]{geometry}
\usepackage[caption=false,font=footnotesize]{subfig}

\AtBeginDocument{\begin{CJK*}{UTF8}{song}\CJKtilde\CJKindent\CJKcaption{utf8}}
\AtEndDocument{\end{CJK*}}

\setlength{\parskip}{0.75ex plus .2ex minus .5ex}
\renewcommand{\baselinestretch}{1.2}

\hypersetup {
    pdftitle={无线传感器网络中的位置相关安全},
    pdfauthor={杨文博}
}

\title{无线传感器网络中的位置相关安全}
\author{杨文博}

\begin{document}

\maketitle

\section{项目的立项依据} 

(研究意义、国内外研究现状及发展动态分析,需结合科学研究发展趋势来论述科学意义;或结合国民经济和社会发展中迫切需要解决的关键科技问题来论述其应用前景。附主要参考文献目录)

上世纪末到本世纪初,通信、嵌入式和分布式计算技术有了飞速的发展,同时得益于微机电系统的长足进步,人们研制出各种不同功用的廉价微型传感器。这些传感器可以感知、测量并收集所处的环境信息,经过通信网络传输给传感器的部署者。无线传感器网络就是由大量支持无线通信的廉价传感器节点组成的分布式、自组织的无线网络。它可以广泛使用在国防军事、国家安全、环境检测、火灾预警、仓储物流、交通管理、医疗卫生和灾难救援等许多领域,帮助人们及时获得有价值的目标环境信息,协助管理者做出正确的决策。\cite{Capkun2006a}

在工业应用和技术研发方面,无线传感器网络保持着强劲的增长势头。专注于无线技术的市场调研公司~ON World~在~2007~年~9~月发布的报告《无线传感器网络产业》中预计到~2011~年,世界市场无线传感器网络系统与服务价值将升至约~46亿~美元,比~07~年增长~5~亿美元~\footnote{ON World Research: \href{http://onworld.com/smartindustries/}{WSN for Smart Industries - A Market Dynamics Report }};市场研究机构~WTRS (West Technology
Research Solutions)~在~2008~年~12~月发布的《无线传感器网络技术趋势报告》中预测可用于无线传感器网络的低功耗~WiFi~芯片市场将在~2008~到~2013~年间保持~322\%~的复合年均增长率~\footnote{WTRS: \href{http://www.wtrs.net/wcntechtrends.htm}{Wireless Sensor Network Technology Trends Q4 2008}};ON World~公司~2009~年~1~月发布的《无线传感器网络研发趋势》报告预测到~2012~年,全球无线传感器网络研发支出将达到~13~亿美元,是~2007~年的~2.5~倍~\footnote{ON World Research: \href{http://onworld.com/wsn/}{Wireless Sensor Networks — R\&D Trends and Funding Opportunities }};

无线传感器节点一般情况下由感应模块、处理器、内存、能量供应、无线模块和控制单元组成。装配着不同的感应模块的传感器可以感知和测量物理环境的不同信息,例如湿度、温度、压力、震动、风速、声音、辐射、有毒气体含量等等,因而可以应用于不同的环境中。由于其廉价性和微型化,无线传感器所采用的处理器一般比较低端,不支持如浮点运算、多媒体指令等一些高级功能。它们一般都只配备少量的内存,所收集到的信息将使用无线方式传输到基站。一般情况下,无线传感器节点的能量主要由电池供应,根据环境的不同可以采用太阳能等其它的能量供应方式作为后备能源。

典型的无线传感器网络一般由数十到数千个无线传感器节点组成,由人工或者机械撒播在目标区域,用来检测一定范围区域内的环境信息。受无线传感器节点体积小、数量大、资源受限的限制,无线传感器网络往往具有以下特点:

\begin{enumerate}

\item 树形路由、多跳转发。无线传感器节点需要将收集到的信息传回基站,所以一般构成以基站为根节点的树形结构;由于信号的覆盖范围受限,无线传感器节点间通信往往需要经过多跳转发,其转发由传感器节点完成,没有专门的路由设备。

\item 通信的带宽、稳定性和安全性较差。由于底层采用无线通信,受到信道本身物理特性的限制,~WSN~的通信质量和稳定性往往较差;考虑到无线信号的开放性,其更容易受到信道窃听、伪装、拒绝服务等攻击,需要特别考虑一些安全需求。

\item 网络资源受限。~WSN~中,无线节点往往不具有长期的电源供应,节点设计的计算能力、存储空间都要比一般的有线网络节点要小得多。在设计网络网络结构时需要特别考虑到能耗因素,避免部分节点能源耗尽导致整个网络失灵。

\end{enumerate}


\section{项目的研究内容、研究目标,以及拟解决的关键科学问题} 

(此部分为重点阐述内容)

\section{拟采取的研究方案及可行性分析} 

(包括有关方法、技术路线、实验手段、关键技术等说明)

\section{本项目的特色与创新之处} 

()

\bibliographystyle{IEEEtran}
\bibliography{IEEEfull,wsn}

\end{document}

